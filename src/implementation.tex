\section{Cellular Complexes with GraphBLAS }\label{graphblas-implementation}
%=================================================================

We have implemented in Julia~\cite{BEKS14}---the novel language for scientific computing---our topological operations over cell complexes, using the package \href{https://github.com/abhinavmehndiratta/GraphBLAS.jl}{\texttt{SuiteSparseGraphBLAS.jl}}, which is a Julia wrapper~\cite{Mehndiratta:2019} for \texttt{SuiteSparse:GraphBLAS}, i.e., the GraphBLAS standard~\cite{osti_1208646,DBLP:journals/corr/KepnerABBFGHKLM16,Buluc:7965104} provided within the \emph{SuiteSparse} library~\cite{Davis:2018}  of sparse matrix software. 


\subsection{Geometric / topological sparse matrices}
%------------------------------------------------

Within a typical computational pipeline in geometric applications, we may distinguish at least three types of sparse matrices: (a) characteristic matrices of cells as subsets of vertices; (b) boundary representations of edges, faces and solid cells; (c) matrix representation of binary incidence/adjacency relations between cells of different dimension. 

\paragraph{Characteristic matrices} provide the simplest representation of the independent elements (i.e., those that cannot be generated by linear combination of other elements) of a $p$-chain space $C_p$ ($0\leq p \leq d$) from a cellular $d$-complex ($2\leq d \leq 3$). They are built from arrays of arrays of vertex indices using the sol-called ``coords'' method, i.e., starting from $(i,j,x)$ triples.

\paragraph{Boundary operators} give a simple mathematical representation of the so-called ``B-reps'', normally used for solid models, in our case extended to cells of every dimension. In particular, every column of the $[\partial_p]$ matrix gives the signed representation of a basis $p$-cell (i.e., an independent $p$-chain) as a ($p$-1)-cycle (i.e., a ($p$-1)-chain without-boundary). They are built by multiplication of two characteristic matrices, followed by suitable ``filtering'' of  values.

\paragraph{Incidence relations} are generated by multiplication of the characteristic matrices of the two types of cells (of dimension $p$ and $q$, say) under consideration. The non-zero $(i,j)$ element of the product matrix provides the ``strength'' of the elementary incidence, i.e., the number of vertices shared between the $i$-th $p$-cell and the $j$-th $q$-cell.  Their building is done by multiplication of two appropriate instances of (co)boundary matrices, as shown in Table~\ref{tab:relations}b.


\subsection{Matrix computation of boundary chain}
%------------------------------------------------

The mathematics used to compute the graded chain complex $C_\bullet = (C_p, \partial_p)$ starting from sparse binary characteristic matrices $M_p$, with $p$-cells indexing the rows and $0$-cells indexing the columns is given below.
The boundary matrices $\partial_p$ ($1\leq p\leq 3$) between non-oriented chain spaces are computed by \emph{sparse matrix multiplication} of characteristic matrices, followed by \emph{matrix filtering},  produced in Julia by broadcasting vectorized integer division, i.e., ``$\texttt{.}\!\div$'', as follows:

{\footnotesize\begin{lstlisting}
$\partial_1 \texttt{ = } \texttt{M}_0 * \texttt{M}_1' \texttt{ = } \texttt{M}_1'$  
$\partial_2 \texttt{ = } (\texttt{M}_1 * \texttt{M}_2')\ \texttt{.}\!\div\ \texttt{sum(}\texttt{M}_1,\texttt{dims=}2)$
$\partial_3 \texttt{ = } (\texttt{M}_2 * \texttt{M}_3')\ \texttt{.}\!\div\ \texttt{sum(}\texttt{M}_2,\texttt{dims=}2)$
\end{lstlisting}}


\subsection{GraphBLAS computation of boundary chain}
%------------------------------------------------

Our current open-source implementation of cellular and chain complexes using sparse matrices and \href{https://github.com/abhinavmehndiratta/GraphBLAS.jl}{\texttt{SuiteSparseGraphBLAS.jl}} is mantained in \href{https://github.com/gmgigi96/SparseMM}{\texttt{https://github.com/gmgigi96/SparseMM}}, where it provides fast and easy-to-use matrix tools for geometric and topological computing, including the input of cellular complexes, the computation of (unsigned) boundary operators, the answer to single and multiple queries about incidence or adjacency of cells.

