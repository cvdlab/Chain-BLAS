\section{Introduction}\label{introduction}
%=========================================

The aim of this paper is to introduce an approach to 2D and 3D
computational topology and geometry using a \emph{GraphBLAS} sparse matrix
representation of \emph{chain complexes} with linear operators $\partial_p$ and $\delta_{p-1} = \partial_p^\top$ between linear \emph{chain} spaces $C_p$:
\[ 
C_\bullet = (C_p, \partial_p) := 
C_3 \ 
\substack{
\delta_2 \\
\longleftarrow \\[-1mm]
\longrightarrow \\
\partial_3 
}
\ C_2 \ 
\substack{
\delta_1 \\
\longleftarrow \\[-1mm]
\longrightarrow \\
\partial_2 
}
\ C_1 \ 
\substack{
\delta_0 \\
\longleftarrow \\[-1mm]
\longrightarrow \\
\partial_1 
}
\ C_0 .
\] 
Let us to remark preliminary that when $p\in \{0,1\}$, a chain complex is a representation of a \emph{graph}, with $[\partial_1]$ the \emph{incidence matrix} between 1-cells (edges) in $C_1$ and 0-cells (vertices) in $C_0$. Also, the multiplication $[\partial_1][\delta_0]=[\partial_1][\partial_1]^t$ gives the \emph{adjacency} matrix of the graph, while the diagonal entries provide the degrees of vertices, defined by the number of incident edges on each vertex. The \emph{GraphBLAS} standard~\cite{GraphBLAS:standard} provides a small set of matrix primitives to compute graph properties that, combined together, allow for easy implementation of fast algorithms on large graphs.

In several areas of geometric and topological computing---including
geo-mapping, building information modeling, medical imaging, CAD and
solid modeling, virtual and augmented reality, finite element modeling
and simulation, etc.---the amount and detail of 2D and/or 3D data
continue to grow. Analogously, the need for an unified approach to
graph algorithms and for unified and simplified interfaces, has been
well intercepted by the GraphBLAS initiative and the related GraphBLAS
standardization effort~\href{http://graphblas.org}{\texttt{graphblas.org}}. We would like to show here that the domain
covering of this standard library on graphs can be greatly extended, to cover the
representation of more general \emph{cellular $d$-complexes} ($1\leq d\leq 3$) and their $p$-skeletons ($0\leq p\leq 3$).

(Co)chain complexes, as well (co)boundary operators, are well-known
basic tools of algebraic topology and homological algebra. In
particular, a \emph{chain complex} is a graded sequence of such linear
operators between graded linear spaces of ``chains''. A $p$-chain can be seen as a subset of a finite set $\texttt{V}$ of Euclidean points whose affine hull has dimension $p$.
Chain spaces and sparse matrices are the components of the Linear Algebraic
Representation (\texttt{LAR})~\cite{Dicarlo:2014:TNL:2543138.2543294}, that is being used for
boundary, decompositive, and enumerative
representations~\cite{Requicha:1980:RRS:356827.356833} of models of rigid solid objects.

Some numerical methods aiming to integrate domain modeling, differential
topology and mathematical modeling with physical simulations were based on chains and
cochains, starting with~\cite{PALMER1995733,Palmer1993}. In particular,
Discrete Exterior Calculus (DEC) with simplicial complexes was
introduced by \cite{Hirani:2003:DEC:959640} and made popular by
\cite{Desbrun:2006:DDF:1185657.1185665,Elcott:2006:BYO:1185657.1185666}.
Finite Element Exterior Calculus 
(FEEC) is an advance in the mathematics of finite element 
methods~\cite{arnold_falk_winther_2006,Arnold:2010,Arnold:2018} that employs differential
complexes to construct stable numerical schemes. The Cell
Method~(CM) is a purely algebraic computational method
for modeling and
simulation~\cite{Tonti:1975,Tonti:2013,Ferretti:2014}
based on boundary/coboundary maps and a direct discrete formulation of
field laws. Our own research in geometrical and physical modeling with
chain and cochain complexes was introduced in
\cite{DiCarlo:2009:DPU:1629255.1629273,ieee-tase,Dicarlo:2014:TNL:2543138.2543294}.

More recently, we provided---using sparse
matrices---an algorithmic pipeline~\cite{TSAS:19} to compute the \emph{arrangement} of the
Euclidean \(d\)-space (\(d=2,3\)), i.e.~the partition of
it into a cellular \(d\)-complex, starting from a collection of (possibly intersecting) cellular
(\(d\)-1)-complexes embedded in \(\E^d\). In~\cite{paoluzzi2019finite} we
have shown that the \emph{atoms} of the Boolean algebra generated by
such space partition correspond one-to-one to the columns of the $[\partial_d]$ matrix
of the boundary operator \(\partial_d : C_d \to C_{d-1}\). This allows for fast
native reconstruction of every solid expression in the \emph{solid algebra} of
\(d\)-space generated by the input terms, usually called Constructive Solid Geometry (CSG)
in solid modeling~\cite{Requicha:1980:RRS:356827.356833}.

The topological background of linear chain spaces and chain complexes is summarized in Section~\ref{chain-complexes}, together with the concepts of boundary and coboundary linear maps between chain spaces.
In Section~\ref{chain-adjacencies-and-incidensies} an operational definition of topological queries through composition of operators corresponding to products of their sparse matrices is discussed.
In Section~\ref{boolean-algebras} an interpretation as a Boolean algebra of the partition of the Euclidean $d$-space produced by a collection of geometric objects is provided.
The implementation of such concepts using GraphBLAS is given in Section~\ref{graphblas-implementation}. Some simple examples of topology computation using linear algebra are shown in  Section~\ref{examples}.
In the Conclusion Section we briefly summarize our main points, state our findings, and propose some possible extension of this approach.
