\section{Background: Chain Complexes}\label{chain-complexes}
%===========================================================

Cellular complexes are largely used in Computer Graphics and in Geometric and Solid Modeling~\cite{Elter:10.1007/978-3-642-78114-8_12}. In particular, they provide the geometric-topological discretization for computer modeling and simulation of physical properties of both manmade and natural objects~\cite{DiCarlo:2009:DPU:1629255.1629273,ieee-tase}.

\subsection{Cell complexes vs Chain complexes}
\label{cell-complexes-vs-chain-complexes}
%--------------------------------------------


A \(p\)-chain can be seen as a subset of \(p\)-cells from a cellular
complex. The space of \(p\)-chains is closed
w.r.t.~addition and product times a scalar from a field. In particular, it is a linear (vector) space.

\paragraph{Cells and Chains}\label{sec:chain}
%------------------------------------------------------------------------------
%--
A $p$-\emph{manifold} is a topological space where each point has a neighborhood
that is homeomorphic to $\E^p$. A \emph{$p$-cell} $\sigma$ ($0\leq p\leq d$) of cellular complexes is piecewise-linear, connected, possibly non convex, $p$-manifold, and not necessarily contractible\footnote{The cells of CW-complexes are contractible to a point. With our LAR representation they may contain internal holes.}. An
$r$-face $\tau$ of a $p$-cell $\sigma$ ($0\leq r\leq p$) is an $r$-cell contained
in the frontier of $\sigma$.


A \emph{$p$-chain} can be seen, with some abuse of language, as a collection of $p$-cells.
The set $C=\oplus\ C_p$ of chains can be given the structure of a graded vector space  by
defining sums of chains with the same dimension, and products times scalars in a
field, with the usual properties.

A \emph{basis} $U_p$  is the set of \emph{independent} (or \emph{elementary}) chains $u_p \in C_p$, given
by singletons of $\Lambda_p$ elements. Every chain $c\in C_p$ is uniquely generated by
a linear combination of the basis with field coefficients. Once  the  basis is fixed, the coordinate
representation of each $\{\sigma_k\} = u_k \in C_p$ is unique. This is an ordered
sequence of coefficients, either from $\{0,1\}$ (unsigned representation) or from
$\{-1,0,+1\}$ (signed representation). With abuse of language, we
often call $p$-cells the independent generators of $C_p$, {i.e.}~the elements of
$U_p$.


\paragraph{Chain and cochain complexes}\label{graded-complexes}
%-------------------------------------------------------------------------------
%
A \emph{graded vector space} is a vector space $V$ expressed as a direct sum  of
spaces $V_k$ indexed by integers in $[0,d]$:
\[ V = \oplus_{k = 0}^d V_k, \qquad [0,d] := \{k\in\N \ |\ 0\leq k\leq d\}.
\]

A linear map $f:V\to W$ between graded vector spaces is called a \emph{graded
map} of degree $p\ $ if $f(V_k) \subset W_{k+p}$.

A \emph{chain complex} is a graded vector  space $V$ furnished with a graded
linear map $\partial : V \to V$ of degree $-1$ which satisfies $\partial^2 = 0$, called
\emph{boundary operator}. In other words, a chain complex
is a sequence of vector spaces $C_k$ and linear maps $\partial_k : C_k \to C_{k-1}$,
such that $\partial_{k-1} \circ\ \partial_{k} = 0$.

A \emph{cochain complex} is a graded vector space $V$ furnished with a graded
linear map $\delta : V \to V$ of degree $+1$ which satisfies $\delta^2 = 0$,
called \emph{coboundary operator}. That is to say, a cochain complex is a
sequence of vector spaces $C^k$ and linear maps $\delta^k : C^k \to C^{k+1}$,
such that $\delta^{k+1} \circ\ \delta^{k} = 0$.
%This duality implies that ... (Antonio !)

Since any linear map $L: V\to W$ between linear spaces induces a dual map $L': 
W' \to V'$ between their duals, any chain complex is associated with a dual 
cochain complex, and viceversa:
\[
(\delta^k \omega) g = \omega (\partial_{k+1} g), \qquad \omega \in C^k, g \in C_{k+1}.
\]
 In a Euclidean space, chain and cochain spaces can be trivially identified~\cite{ieee-tase}, so that we use the $C_p$ notation for both spaces, with $\partial_p: C_p\to C_{p-1}$, and $\delta_p = \partial_{p-1}^\top: C_{p-1}\to C_p$.

